\documentclass[10pt,a4paper,twocolumn,openany]{book}
%\documentclass[10pt,a4paper,openany]{book}

\usepackage[bg-letter]{lib/rpg-book} % Options: bg-a4, bg-letter, bg-full, bg-print, bg-none.
\usepackage[english]{babel}
%\usepackage[utf8]{inputenc}


% Start document
\begin{document}
\fontfamily{ppl}
\selectfont % Set text font
\frontmatter

\rpgMakeCover[
    title = Verbose Magic,
    subtitle = Created by Cassandra de la Cruz-Munoz \\ \today \\ \url{https://github.com/Krozark/RPG-LaTeX-Template}
]


\tableofcontents

% Your content goes here
\mainmatter

\chapter{Introduction}

\section{Concept}
Verbose Magic is a system for creating custom magical spells using natural language.
The more specific you get, the more control you have over the spell.
Omitting details leads to assumptions, discussed later in this book.
Magic has a cost to it, drawing mana from the caster's surroundings.
Additional detail to a spell modifies its mana cost, sometimes increasing it and sometimes decreasing it.

\section{How To Use}
Verbose Magic can be used on its own for a high-magic style campaign, or it can be substituted (with modifications for balance) with existing magic systems in other roleplaying systems.
The first part of the book will be about using this system standalone, and the second part will be about integrating it with other systems.	

\chapter{Spellcasting}
To start casting a spell, a character simply starts describing the spell's effect.
To end casting a spell, a character finishes describing the spell's effect.
All spells must be one or more grammatically correct sentences, and be a complete thought.

\begin{rpg-commentbox}
	A "character" in general in this document can refer both to player characters, as well as characters under the GM's control.
\end{rpg-commentbox}

\section{Mana}
Spells cost mana.
Mana cost tends to go up with more words used.
Some descriptive words lower the mana cost.
Additionally, using more complicated words instead of word phrases can reduce the mana cost.
Mana exists in the world, being more concentrated in more natural areas.
Natural processes produce mana.

\begin{rpg-suggestionbox}
	You can, if you prefer, choose to have innate mana instead of environmental mana, where each person has mana inside of them and draws upon that to cast spells.
\end{rpg-suggestionbox}

\begin{rpg-commentbox}
	As this is a playtest version, it's on the group to decide what numbers sound reasonable for how much mana is in the environment, as well as how much mana specific words cost to cast.
	Eventually, a dictionary will be made that contains recommended mana costs for various commonly used words, based on playtest feedback.
\end{rpg-commentbox}

\section{Languages}
It's best if you use a language the entire group knows.
Using a language some people in the group don't know will bog things down as you will have to translate a good deal, and nuances can be lost in translation.

\begin{rpg-suggestionbox}
	If everyone in the group knows at least two of the same languages, it may be more fun to try casting spells in the language that's not the one you'll be using for normal in-game and out-of-game speech.
\end{rpg-suggestionbox}

\section{Assumptions}
What happens when you don't specify an aspect of a spell?
The assumptions mechanism is meant to clarify that.
For spells you'll cast fairly often, the group should determine ahead of time what the casting assumptions are.
For uncommon spells, it's reasonable to determine that on the fly.
The assumptions mechanic is a composite system; each word has assumptions that come with it, and as words are combined in spells, some assumptions cancel out.

\begin{rpg-commentbox}
	Future versions of this system may include a dictionary that contains a recommended set of assumptions for words commonly used in spells, based on playtest feedback.
\end{rpg-commentbox}

\chapter{Integration}
Using Verbose Magic with another roleplaying system will likely require that the power of the Verbose Magic system be weakened, so as to not encourage all characters to use magic exclusively.

\begin{rpg-commentbox}
	Future versions of this system may include mana scaling rules for integration with various roleplaying systems.
\end{rpg-commentbox}

\section{Ease of Use}
Most roleplaying systems with included magic have prewritten concise spells.
With Verbose Magic, that simply will not do.
Instead, characters must describe what the spells will do.
As such it is best to establish what the equivalent Verbose Magic spell is for any given spell in your chosen system.

\begin{rpg-commentbox}
	Future versions of this system may include sample wordings for spells from other roleplaying systems.
\end{rpg-commentbox}

%\section{Some boxes}
%\begin{rpg-titlebox}{title}
%	rpg-titlebox
%\end{rpg-titlebox}
%
%\begin{rpg-commentbox}
%	rpg-commentbox
%\end{rpg-commentbox}
%
%\begin{rpg-warnbox}
%	you can add some warnings using this box
%\end{rpg-warnbox}
%
%\begin{rpg-suggestionbox}
%	rgp-suggestionbox
%\end{rpg-suggestionbox}
%
%\begin{rpg-quotebox}
%    rpg-quotebox
%\end{rpg-quotebox}
%
%\begin{rpg-examplebox}
%	rgp-examplebox
%\end{rpg-examplebox}
%
%%\newpage % Acts as columbreak because of twocolumn option; for pagebreak use \clearpage
%
%\section{Some tables}
%\header{default rpg-table (2 column)}
%\begin{rpg-table}
%   	\textbf{Table head 1}  & \textbf{Table head 2} \\
%   	Some value  & Some value \\
%   	Some value  & Some value \\
%   	Some value  & Some value
%\end{rpg-table}
%
%% For more columns, you can say \begin{rpg-table}[your options here].
%% For instance, if you wanted three columns, you could say
%% \begin{rpg-table}[XXX]. The usual host of tabular parameters are
%% aailable as well.
%\header{rpg-table with more columns}
%\begin{rpg-table}[XXX]
%    \textbf{Table head 1}  & \textbf{Table head 2} & \textbf{Table head 3}\\
%   	Some value  & Some value & Some value\\
%   	Some value  & Some value & Some value\\
%   	Some value  & Some value & Some value
%\end{rpg-table}
%
%\header{default rpg-table2 (2 column)}
%\begin{rpg-table2}
%   	\textbf{Table head 1}  & \textbf{Table head 2} \\
%   	Some value  & Some value \\
%   	Some value  & Some value \\
%   	Some value  & Some value
%\end{rpg-table2}
%
%\header{rpg-longtable}
%\tablehead{%
%    \hline
%    \textbf{first column} & \textbf{second column} & \textbf{third column} \\
%    \hline
%}
%\begin{rpg-longtable}[ccc]
%    1 & 2 & 3 \\ 1 & 2 & 3 \\ 1 & 2 & 3 \\ 1 & 2 & 3 \\
%    1 & 2 & 3 \\ 1 & 2 & 3 \\ 1 & 2 & 3 \\ 1 & 2 & 3 \\
%    1 & 2 & 3 \\ 1 & 2 & 3 \\ 1 & 2 & 3 \\ 1 & 2 & 3 \\
%    1 & 2 & 3 \\ 1 & 2 & 3 \\ 1 & 2 & 3 \\ 1 & 2 & 3 \\
%    1 & 2 & 3 \\ 1 & 2 & 3 \\ 1 & 2 & 3 \\ 1 & 2 & 3 \\
%    1 & 2 & 3 \\ 1 & 2 & 3 \\ 1 & 2 & 3 \\ 1 & 2 & 3 \\
%    1 & 2 & 3 \\ 1 & 2 & 3 \\ 1 & 2 & 3 \\ 1 & 2 & 3 \\
%    1 & 2 & 3 \\ 1 & 2 & 3 \\ 1 & 2 & 3 \\ 1 & 2 & 3 \\
%    1 & 2 & 3 \\ 1 & 2 & 3 \\ 1 & 2 & 3 \\ 1 & 2 & 3 \\
%    1 & 2 & 3 \\ 1 & 2 & 3 \\ 1 & 2 & 3 \\ 1 & 2 & 3 \\
%    1 & 2 & 3 \\ 1 & 2 & 3 \\ 1 & 2 & 3 \\ 1 & 2 & 3 \\
%    1 & 2 & 3 \\ 1 & 2 & 3 \\ 1 & 2 & 3 \\ 1 & 2 & 3 \\
%    1 & 2 & 3 \\ 1 & 2 & 3 \\ 1 & 2 & 3 \\ 1 & 2 & 3 \\
%    1 & 2 & 3 \\ 1 & 2 & 3 \\ 1 & 2 & 3 \\ 1 & 2 & 3 \\
%    1 & 2 & 3 \\ 1 & 2 & 3 \\ 1 & 2 & 3 \\ 1 & 2 & 3 \\
%\end{rpg-longtable}
%
%\section{List}
%\begin{rpg-list}
%    \item first list item
%    \item second list item
%\end{rpg-list}
%
%
%\section{Monster}
%% You can optionally not include the background by saying
%%\begin{rpg-monsterboxnobg}{monsterboxnob}
%\begin{monsterbox}{rpg-monsterbox}
%	\textit{Small metasyntatic variable (golbinoid), neutral evil}\\
%	\rpghline
%	\basics[%
%	armorclass = 12,
%    hitpoints  = \rpgdice{3d8 + 3},
%	speed      = 50 t
%	]
%	\rpghline%
%	\stats[ % This stat command will autocomplete the modifier for you
%    STR = 12, 
%    DEX = 7
%	]
%	\rpghline%
%	\details[%
%	% If you want to use commas in these sections, enclose the
%	% description in braces.
%	% I'm so sorry.
%	languages = {Common Lisp, Erlang},
%	]
%	\rpghline%
%	\begin{rpg-monsteraction}[rpg-monsteraction]
%		This Monster has some serious superpowers!
%	\end{rpg-monsteraction}
%
%	\rpgmonstersection{rpgmonstersection}
%	\begin{rpg-monsteraction}[rpg-monsteraction]
%		This one can generate tremendous amounts of text! Though only when it wants to.
%	\end{rpg-monsteraction}
%
%	\begin{rpg-monsteraction}[rpg-monsteraction]
%    See, here he goes again! Yet more text.
%	\end{rpg-monsteraction}
%\end{monsterbox}
%
%
%\section{Spell}
%\begin{rpg-spell}
%	{Beautiful Typesetting}
%	{4th-level illusion}
%	{1 action}
%	{5 feet}
%	{S, M (ink and parchment, which the spell consumes)}
%	{Until dispelled}
%	You are able to transform a written message of any length into a beautiful scroll. All creatures within range that can see the scroll must make a wisdom saving throw or be charmed by you until the spell ends.
%
%	While the creature is charmed by you, they cannot take their eyes off the scroll and cannot willingly move away from the scroll. Also, the targets can make a wisdom saving throw at the end of each of their turns. On a success, they are no longer charmed.
%\end{rpg-spell}
%
%\chapter{Colors}
%
%This package provides several global color variables to style \lstinline!rpg-commentbox!, \lstinline!rpg-quotebox!, \lstinline!rpg-examplebox!, and \lstinline!rpg-table! environments.
%
%
%TODO

%\begin{rpgtable}[lX]
%  \textbf{Color}         & \textbf{Description} \\
%  \lstinline!commentboxcolor! & Controls \lstinline!commentbox! background. \\
%  \lstinline!paperboxcolor!   & Controls \lstinline!paperbox! background. \\
%  \lstinline!quoteboxcolor!   & Controls \lstinline!quotebox! background. \\
%  \lstinline!tablecolor!      & Controls background of even \lstinline!rpgtable! rows. \\
%\end{rpgtable}
%
%See Table~\ref{tab:colors} for a list of accent colors that match the core books.
%
%\begin{table*}
%  \begin{rpgtable}[XX]
%    \textbf{Color}                            & \textbf{Description} \\
%    \lstinline!commentgreen!                      & Light green used in PHB Part 1 \\
%    \lstinline!commentblue!                       & Light cyan used in PHB Part 2 \\
%    \lstinline!PhbMauve!                           & Pale purple used in PHB Part 3 \\
%    \lstinline!PhbTan!                             & Light brown used in PHB appendix \\
%    \lstinline!DmgLavender!                        & Pale purple used in DMG Part 1 \\
%    \lstinline!DmgCoral!                           & Orange-pink used in DMG Part 2 \\
%    \lstinline!DmgSlateGray! (\lstinline!DmgSlateGrey!) & Blue-gray used in PHB Part 3 \\
%    \lstinline!DmgLilac!                           & Purple-gray used in DMG appendix \\
%  \end{rpgtable}
%  \caption{Colors supported by this package}%
%  \label{tab:colors}
%\end{table*}
%
%\begin{itemize}
%  \item Use \lstinline!\setthemecolor[<color>]! to set \lstinline!themecolor!, \lstinline!commentcolor!, \lstinline!paperboxcolor!, and \lstinline!tablecolor! to a specific color.
%  \item Calling \lstinline!\setthemecolor! without an argument sets those colors to the current \lstinline!themecolor!.
%  \item \lstinline!commentbox!, \lstinline!rpgtable!, \lstinline!paperbox!, and \lstinline!quoteboxcolor! also accept an optional color argument to set the color for a single instance.
%\end{itemize}
%
%\subsection{Examples}
%
%\subsubsection{Using \lstinline!themecolor!}
%
%\begin{lstlisting}
%\setthemecolor[PhbMauve]
%
%\begin{paperbox}{Example}
%  \lipsum[2]
%\end{paperbox}
%
%\setthemecolor[PhbLightCyan]
%
%\header{Example}
%\begin{rpg-table}[cX]
%  \textbf{d8} & \textbf{Item} \\
%  1           & Small wooden button \\
%  2           & Red feather \\
%  3           & Human tooth \\
%  4           & Vial of green liquid \\
%  6           & Tasty biscuit \\
%  7           & Broken axe handle \\
%  8           & Tarnished silver locket \\
%\end{rpg-table}
%\end{lstlisting}
%
%\begingroup
%\setthemecolor[PhbMauve]
%
%\begin{paperbox}{Example}
%  \lipsum[2]
%\end{paperbox}
%
%\setthemecolor[PhbLightCyan]
%
%\header{Example}
%\begin{rpgtable}[cX]
%  \textbf{d8} & \textbf{Item} \\
%  1           & Small wooden button \\
%  2           & Red feather \\
%  3           & Human tooth \\
%  4           & Vial of green liquid \\
%  6           & Tasty biscuit \\
%  7           & Broken axe handle \\
%  8           & Tarnished silver locket \\
%\end{rpgtable}
%\endgroup
%
%\subsubsection{Using element color arguments}
%
%\begin{lstlisting}
%\begin{rpgtable}[cX][DmgCoral]
%  \textbf{d8} & \textbf{Item} \\
%  1           & Small wooden button \\
%  2           & Red feather \\
%  3           & Human tooth \\
%  4           & Vial of green liquid \\
%  6           & Tasty biscuit \\
%  7           & Broken axe handle \\
%  8           & Tarnished silver locket \\
%\end{rpgtable}
%\end{lstlisting}
%
%\begin{rpgtable}[cX][DmgCoral]
%  \textbf{d8} & \textbf{Item} \\
%  1           & Small wooden button \\
%  2           & Red feather \\
%  3           & Human tooth \\
%  4           & Vial of green liquid \\
%  6           & Tasty biscuit \\
%  7           & Broken axe handle \\
%  8           & Tarnished silver locket \\
%\end{rpgtable}



% End document
\end{document}
